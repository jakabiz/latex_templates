\documentclass[10pt,a4paper,oneside,fleqn]{article}
\usepackage[slovene]{babel}     % za prikaz šumnikov
\usepackage[utf8]{inputenc}     % za prikaz šumnikov
\usepackage{amsfonts}           % doda zbirko matematičnih znakov
\usepackage{graphicx}           % prikaz slik
\usepackage{longtable}          % paket za tabele, ki se raztezajo čez več strani
\usepackage{caption}
\usepackage{subcaption}

% Nastavitve strani
\setlength{\oddsidemargin}{-0. cm} 
\setlength{\evensidemargin}{-0. cm}
\setlength{\topmargin}{-0.54 cm}   
\setlength{\textwidth}{16. cm}     
\setlength{\textheight}{24 cm}     
\setlength{\marginparsep}{3 mm}    
\setlength{\marginparwidth}{1.5 cm}

% set table and figure captions
\captionsetup[table]{skip=10pt,singlelinecheck=false}
\captionsetup[figure]{justification=centering}

% Nastavitve odstavka
\setlength{\parindent}{0 mm}       
\setlength{\parskip}{1.5ex plus 0.5ex minus 0.5ex}
\renewcommand{\baselinestretch}{1.25}
\setlength{\mathindent}{1 cm}  
\newcommand{\tabitem}{~~\llap{\textbullet}~~} % item in a table    

% Literatura
\bibliographystyle{IEEEtran_slo}		% literatura je sortirana po vrstnem redu prve navedbe

% Kazalo
\setcounter{tocdepth}{2}

% Povezave, aktiven dokument
\usepackage[unicode]{hyperref}		% naredi dokument aktiven (v DVI in PDF) -> mora biti vedno zadnji pred začetkom dokumenta
%%%%%%%%%%%%%%%%%%%%%%%%%%%%%%%%%%%%%%%%%%%%%%%%%%%%%%%%%%%%%%%%%%%%%%%%%%%%%%%%%%%%%%%%%%%%%%%%%%%%
% sedaj začnemo dokument
\begin{document}

% Naslovnica
\begin{titlepage}
\begin{center}
\begin{figure}[!hb]
\centering
\includegraphics[scale=0.4]{slike/FS_logo}
\end{figure}
\Large{Magistrski študijski program 2. stopnje\\
Strojništvo - Razvojno raziskovalni program}\\
[2cm]
\Large\textbf{PROJEKTNI PRAKTIKUM - MAG}
\\[0.5cm]
\Large\textbf{Poročilo}
\\[2cm]
\end{center}
Študent: {\large Ime Priimek} \hfill vpisna številka: {\large 2 3 x x x x x x}
\\[1cm]
Naslov teme \textit{Projektnega praktikuma - MAG}: 
\\[0.5cm]
Naziv laboratorija/podjetja: 
\\[1cm]
Mentor v laboratoriju FS: 
\\[0.5cm]
Mentor v podjetju: 
\\[2.5cm]
Kraj in datum: Ljubljana, \today
\end{titlepage}


%Kazalo
\tableofcontents
\pagebreak
\listoffigures
\listoftables
\pagebreak

%Jedro
\section{Uvod}

\subsection{Opredelitev projektnega problema}
Lorem ipsum dolor sit amet, consectetur adipiscing elit, sed do eiusmod tempor incididunt ut labore et dolore magna aliqua. Ut enim ad minim veniam, quis nostrud exercitation ullamco laboris nisi ut aliquip ex ea commodo consequat. Duis aute irure dolor in reprehenderit in voluptate velit esse cillum dolore eu fugiat nulla pariatur. Excepteur sint occaecat cupidatat non proident, sunt in culpa qui officia deserunt mollit anim id est laborum.\\
Lorem ipsum dolor sit amet, consectetur adipiscing elit, sed do eiusmod tempor incididunt ut labore et dolore magna aliqua. Ut enim ad minim veniam, quis nostrud exercitation ullamco laboris nisi ut aliquip ex ea commodo consequat. Duis aute irure dolor in reprehenderit in voluptate velit esse cillum dolore eu fugiat nulla pariatur. Excepteur sint occaecat cupidatat non proident, sunt in culpa qui officia deserunt mollit anim id est laborum \cite{SURS}.

\subsection{Namen in cilji}
Lorem ipsum dolor sit amet, consectetur adipiscing elit, sed do eiusmod tempor incididunt ut labore et dolore magna aliqua. Ut enim ad minim veniam, quis nostrud exercitation ullamco laboris nisi ut aliquip ex ea commodo consequat. Duis aute irure dolor in reprehenderit in voluptate velit esse cillum dolore eu fugiat nulla pariatur. Excepteur sint occaecat cupidatat non proident, sunt in culpa qui officia deserunt mollit anim id est laborum.

\subsection{Metode dela}
Lorem ipsum dolor sit amet, consectetur adipiscing elit, sed do eiusmod tempor incididunt ut labore et dolore magna aliqua. Ut enim ad minim veniam, quis nostrud exercitation ullamco laboris nisi ut aliquip ex ea commodo consequat. Duis aute irure dolor in reprehenderit in voluptate velit esse cillum dolore eu fugiat nulla pariatur. Excepteur sint occaecat cupidatat non proident, sunt in culpa qui officia deserunt mollit anim id est laborum.

\pagebreak

\section{Merilna proga in merilna oprema}
Lorem ipsum dolor sit amet, consectetur adipiscing elit, sed do eiusmod tempor incididunt ut labore et dolore magna aliqua. Ut enim ad minim veniam, quis nostrud exercitation ullamco laboris nisi ut aliquip ex ea commodo consequat. Duis aute irure dolor in reprehenderit in voluptate velit esse cillum dolore eu fugiat nulla pariatur. Excepteur sint occaecat cupidatat non proident, sunt in culpa qui officia deserunt mollit anim id est laborum \cite{EN_60436}. 

\begin{figure}[h]
\centering
\includegraphics[width=0.8\textwidth]{slike/PPA_datalogger}
\caption{Posnetek zaslona programa PPA Datalogger za upravljanje analizatorja moči.}\label{fig:PPA_datalogger}
\end{figure}

\subsection{Podpoglavje}\label{}
Lorem ipsum dolor sit amet, consectetur adipiscing elit, sed do eiusmod tempor incididunt ut labore et dolore magna aliqua. Ut enim ad minim veniam, quis nostrud exercitation ullamco laboris nisi ut aliquip ex ea commodo consequat. Duis aute irure dolor in reprehenderit in voluptate velit esse cillum dolore eu fugiat nulla pariatur. Excepteur sint occaecat cupidatat non proident, sunt in culpa qui officia deserunt mollit anim id est laborum.

Glede na dokumentacijo analizatorja moči \cite{PPA_manual} se določi merilno negotovost toka (enačba \ref{eq:tok}), napetosti (enačba \ref{eq:napetost}) in moči (enačba \ref{eq:moč}) pri meritvah na izmeničnem toku:

\begin{equation}\label{eq:tok}
u(I) = 0,0001 \cdot Rdg + 0,00038 \cdot Rng + \left(0,00004 \cdot \frac{f}{1000} \cdot Rdg \right) + 30 \cdot 10^{-6}
\end{equation}

\begin{equation}\label{eq:napetost}
u(U) = 0,0001 \cdot Rdg + 0,00038 \cdot Rng + \left(0,00004 \cdot \frac{f}{1000} \cdot Rdg \right) + 5 \cdot 10^{-3}
\end{equation}

\begin{equation}\label{eq:moč}
u(P) = \left(0,0002 + \frac{0,0003}{pf} + 0,00005 \frac{f}{1000 \cdot pf} \right) \cdot Rdg + VA Rng 
\end{equation}

pri čemer so:\\
\begin{tabular}{l l}
\tabitem $u(I)$ & merilna negotovost električnega toka v A,\\
\tabitem $u(U)$ & merilna negotovost električne napetosti v V, \\
\tabitem $u(P)$ & merilna negotovost električne moči v W, \\
\tabitem $Rdg$ & odčitek merjene veličine, \\
\tabitem $Rng$ & merilno območje merjene veličine, \\
\tabitem $f$ & frekvenca v Hz, \\
\tabitem $pf$ & faktor moči in \\
\tabitem $VA Rng$ & merilno območje spremenljivke $UA$ v VA.
\end{tabular}

Lorem ipsum dolor sit amet, consectetur adipiscing elit, sed do eiusmod tempor incididunt ut labore et dolore magna aliqua. Ut enim ad minim veniam, quis nostrud exercitation ullamco laboris nisi ut aliquip ex ea commodo consequat. Duis aute irure dolor in reprehenderit in voluptate velit esse cillum dolore eu fugiat nulla pariatur. Excepteur sint occaecat cupidatat non proident, sunt in culpa qui officia deserunt mollit anim id est laborum.

\begin{table}[h]
    \caption{Meritve rabe električne energije in porabe vode pri različnih pogojih programa hitro pranje.}
    \label{tab:meritve_2}
    \begin{tabular}{l l l l l l}
    \hline
        program & pogoji & $m_v$ [kg] & $t$ [s] & $E$ [Wh] & $u(E)$ [Wh] \\ \hline
        hitro pranje & prazen stroj & 11,378 & 4260 & 813,75 & 0,56 \\ 
        hitro pranje & prazen stroj, funkcija 1/2 & 11,526 & 4140 & 792,13 & 0,53 \\ 
        hitro pranje & poln stroj & 11,374 & 4381 & 973,35 & 0,64 \\ 
        hitro pranje & poln stroj, funkcija 1/2 & 11,378 & 4260 & 929,21 & 0,61 \\ 
        hitro pranje & poln stroj, funkcija tablet & 11,246 & 4561 & 1008,50 & 0,69 \\ \hline
    \end{tabular}
\end{table}

\begin{figure}[h]

\begin{subfigure}{0.6\textwidth}
\includegraphics[width=1\linewidth]{slike/skatla_1} 
\caption{}
\label{fig:subim1}
\end{subfigure}
\begin{subfigure}{0.4\textwidth}
\includegraphics[width=1\linewidth]{slike/skatla_2}
\caption{}
\label{fig:subim2}
\end{subfigure}

\caption{(a) - varovalna škatlica s priključki na volmeter in ampermeter (PPA5500), (b) - električna shema varovalne škatlice.}
\label{fig:skatla}
\end{figure}

\pagebreak

\section{Zaključek}
Lorem ipsum dolor sit amet, consectetur adipiscing elit, sed do eiusmod tempor incididunt ut labore et dolore magna aliqua. Ut enim ad minim veniam, quis nostrud exercitation ullamco laboris nisi ut aliquip ex ea commodo consequat. Duis aute irure dolor in reprehenderit in voluptate velit esse cillum dolore eu fugiat nulla pariatur. Excepteur sint occaecat cupidatat non proident, sunt in culpa qui officia deserunt mollit anim id est laborum. Te spremenljivke so:
\begin{itemize}
\item čas trajanja programa,
\item količina porabljene vode,
\item temperatura programa in
\item produkt mase in specifične toplote posode.
\end{itemize}
Lorem ipsum dolor sit amet, consectetur adipiscing elit, sed do eiusmod tempor incididunt ut labore et dolore magna aliqua. Ut enim ad minim veniam, quis nostrud exercitation ullamco laboris nisi ut aliquip ex ea commodo consequat. Duis aute irure dolor in reprehenderit in voluptate velit esse cillum dolore eu fugiat nulla pariatur. Excepteur sint occaecat cupidatat non proident, sunt in culpa qui officia deserunt mollit anim id est laborum.

\pagebreak

%Literatura
\addcontentsline{toc}{section}{Literatura}
\bibliography{PP4_Literatura}

\end{document} 